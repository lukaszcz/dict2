\documentclass[a4paper,12pt]{article}
\usepackage[OT4]{fontenc}
\usepackage[utf8]{inputenc}
\usepackage{a4wide}
\usepackage{longtable}

\begin{document}

\title{The format of cache files}
\author{Łukasz Czajka}
\date{\today}
\maketitle

\newpage

\tableofcontents

\newpage

\section{Basic types}

The following table gives a short description of basic (atomic) types.

In all the following tables it is assumed that \verb#sizeof(int) == 4#
and \verb#sizeof(void *) == 4#. If this is not the case, then the
sizes and offsets should be changed accordingly.

When two numbers x/y appear for a size, then x is for 32-bit, y for
64-bit architectures.


\begin{longtable}{|p{0.5in}|p{0.5in}|p{1.7in}|p{2.8in}|}
\hline
{\bf Name} & {\bf Size} & {\bf Equivalent C type} & {\bf Description}\\
\hline
\endhead

uint & 4 & \verb#unsigned int# & unsigned integer

\\
\hline

int & 4 & \verb#int# & signed integer

\\
\hline

foff & 4/8 & \verb#void *# & an offset from the beginning of the cache file

\\
\hline

short & 2 & \verb#unsigned short int# & a short unsigned integer

\\
\hline
\caption{Basic types}
\end{longtable}



\section{General layout}

Each cache file stores one dictionary (\verb#dict_t#) together with its
hashtable.

All offsets are from the beginning of parent components.

The following table gives a general layout of the file. The components
are in the order they are written to the file by the
application. However, they may appear in any order, save that \verb#Header#
must always be first.

\begin{longtable}{|p{0.6in}|p{2.3in}|p{2.7in}|}
\hline
{\bf Name} & {\bf Size} & {\bf Description}\\
\hline
\endhead

\verb#Header# & 24/28 & The header contains all data necessary to locate other
components.

\\
\hline

\verb#Table# & \verb#Header->tablelength * 4/8# & The table containing
file offsets of \verb#Entry# components. It mirrors
\verb#dict->hash->table#.

\\
\hline

\verb#Entries# & \verb#Header->entrycount *# \verb#sizeof(Entry)# & A group of all the
\verb#Entry# components.

\\
\hline

\verb#Lists# & --- & A group of all the \verb#List# components.

\\
\hline
\caption{Main components of a cache file}
\end{longtable}



\section{Header}


\begin{longtable}{|p{1in}|p{0.6in}|p{0.6in}|p{0.6in}|p{2.7in}|}
\hline
{\bf Name} & {\bf Offset} & {\bf Size} & {\bf Type} & {\bf Description}\\
\hline
\endhead

\verb#dict->size# & 0 & 4 & uint & The size of the dictionary stored --
the number of lines in the dictionary file.

\\
\hline

\verb#tablelength# & 4 & 4 & uint & The length of the hashtable.

\\
\hline

\verb#entrycount# & 8 & 4 & uint & The number of entries in the hashtable --
the number of \verb#Entry# components.

\\
\hline

\verb#loadlimit# & 12 & 4 & uint &

\\
\hline

\verb#primeindex# & 16 & 4 & uint &

\\
\hline

\verb#hashtab_off# & 20 & 4/8 & foff & The file offset of \verb#Table#.

\\
\hline
\caption{Header}
\end{longtable}



\section{Table}

The table contains file offsets of \verb#Header->tablelength# \verb#Entry#
components, which are the heads of hashtable buckets.
\medskip

\section{Entry}

\begin{longtable}{|p{1in}|p{0.6in}|p{0.6in}|p{0.6in}|p{2.7in}|}
\hline
{\bf Name} & {\bf Offset} & {\bf Size} & {\bf Type} & {\bf Description}\\
\hline
\endhead

\verb#s_off# & 0 & 4 & uint & see \verb#hashtable_private.h#

\\
\hline

\verb#s_len# & 4 & 4 & uint & see \verb#hashtable_private.h#

\\
\hline

\verb#h# & 8 & 4 & uint & see \verb#hashtable_private.h#

\\
\hline

\verb#v_off# & 12 & 4/8 & foff & The offset of the List component associated
with this \verb#Entry#.

\\
\hline

\verb#next_off# & 16/20 & 4/8 & foff & The offset of the next \verb#Entry# in the
current bucket, or 0 if this is the last \verb#Entry# in a
bucket. Entries in the same bucket should be stored at consecutive
positions for performance reasons.

\\
\hline
\caption{Entry}
\end{longtable}


\section{List}

\verb#List# is a sequence of zero-terminated integers, i.e. it
consists of a number of non-zero integers and a zero integer
terminator. The integers represent entry line indicies (see
\verb#dict.h#, \verb#list.h#).

\end{document}
