\documentclass[a4paper,12pt]{article}
\usepackage[OT4]{fontenc}
\usepackage[utf8]{inputenc}
\usepackage{a4wide}
\usepackage{longtable}

\begin{document}

\title{The Word Formation Algorithm}
\author{Łukasz Czajka}
\date{\today}
\maketitle

\newpage

\tableofcontents

\newpage

\section{Problem statement}

Given a string \verb#w# representing a word in a language \verb#L#,
we wish to obtain all words in \verb#L# that may be derived from
\verb#w# according to some predefined rules \verb#R# for that language.

For instance, if the language is German and \verb#w = "machen"#, we
would like to obtain ``gemacht'', ``machte'', etc. The set of rules
\verb#R# should contain rules for inflecting words and forming
other derivatives.

\section{Rules}

The algorithm actually uses several sets of rules for one
language. Rules for ``inflect'', ``stem'' and ``forms'' options are
stored in separate files on disk.

\end{document}
